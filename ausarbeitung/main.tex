\documentclass[12pt,a4paper]{article}

\usepackage{graphicx} % Required for inserting images
\usepackage{geometry}
\usepackage{mathptmx} % Closest you can get to Times New Roman
\usepackage[ngerman]{babel} % German language support
\usepackage{footmisc}
\usepackage{microtype}

% Anpassung der Schriftgröße der Fußnoten
\renewcommand{\footnotesize}{\fontsize{10pt}{12pt}\selectfont}

\usepackage{ragged2e} % Text formatting
\justifying % justify text


% Zeilenabstand
\usepackage{setspace}
\setstretch{1.5}

% Seitenabstände
\geometry{top=3cm, left=3cm, right=3cm, bottom=3cm}

\graphicspath{ {img/} }

% No indent für ganzes dokument
\setlength\parindent{0pt}


\title{Inwiefern unterscheiden sich n8n-basierte Workflows mit ChatGPT-Agenten und Ollama-Agenten in der Genauigkeit und Effizienz bei der Erkennung von Fake News in Nachrichtenschlagzeilen?
\\ \large Wissenschaftliche Themenarbeit}

\author{Jonas Schierhold, Henning Franzen, Jan Osing}
\date{Januar 2026}

\begin{document}

    \maketitle
    \thispagestyle{empty}

    \begin{center}
        \includegraphics[width=150pt]{img/IHK-Logo.jpg} \includegraphics[width=150pt]{img/WHS-Logo.png}
    \end{center}

    \vfill
    \noindent
    \begin{tabular}{p{4cm}l}
        Matrikelnummern: & \#\#\# \tabularnewline
        Studiengang: & Duales Studium Bachelor of Arts Wirtschaftsinformatik \tabularnewline
        Gruppe: & IT-BW-18 \tabularnewline
        Themensteller: & Prof. Dr. Kruse \tabularnewline
        Abgabedatum: & 27. Februar 2026 \tabularnewline
    \end{tabular}

    \newpage
    \pagenumbering{Roman} % Setzt die Nummerierung auf römnische Zahlen

    \tableofcontents

    \newpage


    \section*{Abkürzungsverzeichnis}

    \noindent
    \begin{tabular}{p{4cm}l}
        WWW & World Wide Web \tabularnewline
    \end{tabular}

    \newpage
    \pagenumbering{arabic} % Wechsel zur arabischen Nummerierung
    \setcounter{page}{1} % Startet die Nummerierung bei 1
    \setcounter{section}{0} % Setzt den Abschnittszähler auf 0


    \section{Einleitung}
    Seit einigen Jahren verändert sich das Diskussionsklima in Deutschland. Der Umgangston wird rauer, die Meinungen
    polarisieren und die Bereitschaft, einer Position der Gegenseite inhaltlich zuzustimmen, nimmt ab.
    \par
    Besonders durch schnelllebige Social Media Formate wie TikTok oder Instagram-Reels ist es einfach geworden,
    Falschaussagen zu verbreiten, um die eigene Position zu stärken. Botnetzwerke verbreiten gezielt Posts und
    Kommentare mit Falschinformationen und Verschwörungsmythen, die von der vermeintlichen Klimalüge bis zu extrem
    verzerrten oder historisch falschen Narrativen reichen.
    \par
    Gleichzeitig nimmt die Nutzung von generativer KI an Fahrt auf. 2025 nutzten 65\% der befragten laut eigener Angabe
    KI regelmäßig. Das ist ein deutliche Anstieg, denn zwei Jahre vorher waren es noch 37\%.
    \par
    Vor diesem Hintergrund liegt der Einsatz generativer KI zur Überprüfung von Informationen nahe. Dabei stellt sich
    jedoch die Frage nach der Zuverlässigkeit und Effektivität, welche wir im Folgenden bei verschiedenen Modellen
    vergleichen wollen.



    \section{Technologische Grundlagen}
    n8n


    \section{Methodisches Vorgehen}
    Hier kommt die Beschreibung des Methodikus.


    \section{Prototypische Implementierung}
    Hier kommt die Beschreibung der Prototypischen Implementierung.


    \section{Auswertung}
    Hier kommt die Beschreibung und Auswertung.


    \section{Fazit}
    Hier kommt das Fazit.

    \newpage

    \pagenumbering{Roman} % Setzt die Nummerierung auf römnische Zahlen
    \setcounter{page}{3} % Setzt die Nummerierung auf 3

    \begin{thebibliography}{99}
        \bibitem{Grisse2017} \textit{Grisse, Karina}, Was bleibt von der Störerhaftung?, GRUR 2017, S. 1073-1075

    \end{thebibliography}


\end{document}
