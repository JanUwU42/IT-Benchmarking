\documentclass[12pt,a4paper]{article}

\usepackage{graphicx} % Required for inserting images
\usepackage{geometry}
\usepackage{footmisc}
\usepackage{microtype}

\usepackage{helvet} % Set font to Arial
\renewcommand{\familydefault}{\sfdefault} % Set font to Arial

\usepackage[ngerman]{babel} % German language support

\usepackage
[backend=biber,style=apa,sorting=nyt]{biblatex}
\addbibresource{main.bib}

% Anpassung der Schriftgröße der Fußnoten
\renewcommand{\footnotesize}{\fontsize{10pt}{12pt}\selectfont}

\usepackage{ragged2e} % Text formatting
\justifying % justify text


% Zeilenabstand
\usepackage{setspace}
\setstretch{1.5}

% Seitenabstände
\geometry{
    top=3 cm,
    left=6 cm,
    right=2 cm,
    bottom=2.5 cm,
    headsep=1 cm
}

\graphicspath{ {img/} }

% No indent für ganzes dokument
\setlength\parindent{0pt}


\title{Inwiefern unterscheiden sich n8n-basierte Workflows mit ChatGPT-Agenten und Ollama-Agenten in der Genauigkeit und Effizienz bei der Erkennung von Fake News in Nachrichtenschlagzeilen?
\\ \large Wissenschaftliche Themenarbeit}

\author{Jonas Schierhold, Henning Franzen, Jan Osing}
\date{Januar 2026}

\begin{document}

    \maketitle
    \thispagestyle{empty}

    \begin{center}
        \includegraphics[width=150pt]{img/IHK-Logo.jpg} \includegraphics[width=150pt]{img/WHS-Logo.png}
    \end{center}

    \vfill
    \noindent
    \begin{tabular}{p{4cm}l}
        Matrikelnummern: & \#\#\# \tabularnewline
        Studiengang: & Duales Studium Bachelor of Arts Wirtschaftsinformatik \tabularnewline
        Gruppe: & IT-BW-18 \tabularnewline
        Themensteller: & Prof. Dr. Kruse \tabularnewline
        Abgabedatum: & 27. Februar 2026 \tabularnewline
    \end{tabular}

    \newpage
    \pagenumbering{Roman} % Setzt die Nummerierung auf römnische Zahlen

    \tableofcontents

    \newpage


    \section*{Abkürzungsverzeichnis}

    \noindent
    \begin{tabular}{p{4cm}l}
        WWW & World Wide Web \tabularnewline
    \end{tabular}

    \newpage
    \pagenumbering{arabic} % Wechsel zur arabischen Nummerierung
    \setcounter{page}{1} % Startet die Nummerierung bei 1
    \setcounter{section}{0} % Setzt den Abschnittszähler auf 0


    \section{Einleitung}
    Seit einigen Jahren verändert sich das Diskussionsklima in Deutschland. Der Umgangston wird rauer, die Meinungen
    polarisieren und die Bereitschaft, einer Position der Gegenseite inhaltlich zuzustimmen, nimmt
    ab.\footnote{\cite[S.~8]{Bertelsmann}.}


    Besonders durch schnelllebige Social Media Formate wie TikTok oder Instagram-Reels ist es einfach geworden,
    Falschaussagen zu verbreiten, um die eigene Position zu stärken.\footnote{\cite[S.~10]{Nim}.} Botnetzwerke verbreiten
    gezielt Posts und Kommentare mit Falschinformationen und Verschwörungsmythen, die von der vermeintlichen Klimalüge
    bis zu extrem verzerrten oder historisch falschen Narrativen reichen.\footnote{\cite{Bsi}.}

    Gleichzeitig nimmt die Nutzung von generativer KI an Fahrt auf. 2025 nutzten 65\% der Befragten einer
    repräsentativen Studie\footnote{\cite{TV-Verband}.} laut eigener Angabe KI regelmäßig. Das ist ein deutliche Anstieg,
    denn zwei Jahre vorher waren es noch 37\%.

    Vor diesem Hintergrund liegt der Einsatz generativer KI zur Überprüfung von Informationen nahe. Dabei stellt sich
    jedoch die Frage nach der Zuverlässigkeit und Effektivität, welche wir im Folgenden bei verschiedenen Modellen
    vergleichen wollen.

    Für den Vergleich werden wir Benchmarking verwenden. Benchmarking ist...


    \section{Technologische Grundlagen}
    \subsection {n8n}
    n8n ist eine Open-Source Workflow-Automatisierungsplattform, mit der digitale Prozesse automatisiert, verknüpft und gesteuert werden können. n8n verfolgt einen Low-Code/No-Code Ansatz, das heißt, dass Nutzende Workflows visuell erstellen können, ohne umfangreiche Programmierkenntnisse zu benötigen. \footnote{\cite{n8n-Workflow-Automation}.}

    n8n bietet eine Vielzahl von vorgefertigten Integrationen (sogenannte "Nodes") für verschiedene Dienste und Anwendungen, darunter auch native Integrationen mit KI-Tools wie ChatGPT und Ollama. Wenn allerdings die vorgefertigten Nodes nicht ausreichen, bietet n8n die Möglichkeit JavaScript oder Python zu verwenden, um benutzerdefinierte Logiken zu implementieren oder auch eigene Nodes zu bauen. \footnote{\cite{n8n-docs}.}

    \subsection{AI-Benchmarking}
    TODO

    \subsection{KI-Agenten}
    TODO

    \subsection{LLM}
    TODO

    \subsection{Ollama  \& Begrenzung eigene Hardware}
    \subsubsection{Ollama}
    Ollama ist ein Open-Source-Tool, das die lokale Bereitstellung und Ausführung von Large Language Models auf eigener Hardware ermöglicht. Das Tool  ist für den Einsatz in Umgebungen mit begrenzten Rechenressourcen sowie für den Offline-Betrieb ausgelegt und erlaubt die Ausführung leistungsfähiger Sprachmodelle ohne Anbindung an externe Cloud-Infrastrukturen. Da die Verarbeitung vollständig lokal erfolgt, verbleiben sämtliche Daten innerhalb der eigenen Systemumgebung. Auf diese Weise können Anforderungen an Datenschutz, Datensouveränität und Vertraulichkeit zuverlässig eingehalten werden, wodurch sich Ollama insbesondere für Anwendungskontexte mit einem besonderen Fokus auf Datenschutz und IT-Sicherheit eignet. \footnote{\cite{Ollama-Local-Models-Case-Study}.}

    \subsubsection{Begrenzung eigene Hardware}
    TODO

    \section{Methodisches Vorgehen}
    Hier kommt die Beschreibung des Methodikus.


    \section{Prototypische Implementierung}
    Hier kommt die Beschreibung der prototypischen Implementierung.


    \section{Auswertung}
    Hier kommt die Beschreibung und Auswertung.


    \section{Fazit}
    Hier kommt das Fazit.

    \newpage

    \pagenumbering{Roman} % Setzt die Nummerierung auf römnische Zahlen
    \setcounter{page}{3} % Setzt die Nummerierung auf 3

    \printbibliography


\end{document}
